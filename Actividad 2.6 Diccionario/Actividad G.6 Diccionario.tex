\documentclass[conference]{IEEEtran} 
\IEEEoverridecommandlockouts
\usepackage{cite}
\usepackage{amsmath,amssymb,amsfonts}
\usepackage{algorithmic}
\usepackage{flafter}
\usepackage{graphicx}
\usepackage{textcomp}
\usepackage{xcolor}
\graphicspath{ {imagenes/} }

\begin{document}

\title{Actividad 1.1 \\ Sistemas de Numeración }

\author{\IEEEauthorblockN{Ricardo David López Arellano}
\IEEEauthorblockA{\textit{Departamento de Ingeniería en Computación} \\
\textit{CUCEI}\\
Universidad de Guadalajara\\
ricardo.lopez1361@alumnos.udg.mx} }
\onecolumn

\maketitle

\begin{abstract}
El diccionario es un espacio donde es posible crear nuevas palabras, de tal manera que se puede expandir el lenguaje (metalenguaje).
\end{abstract}

\section{Originalidad}
Me comprometo a producir trabajo académico íntegro, lo que significa un trabajo que se adhiere a los estándares intelectuales y académicos de atribución exacta de las fuentes, uso y recolección de datos apropiados, y transparencia en el reconocimiento de las contribuciones de las ideas, descubrimientos, interpretaciones y conclusiones de otros.
Acepto que la trampa en los exámenes, el plagio o la fraudulenta representación de las ideas o lenguaje de otros como propio, la falsificación de datos o cualquier otra instancia de deshonestidad académica, violan los estándares de LA MATERIA, así como los estándares del mundo en general en el campo del conocimiento y las relaciones.

\section{Introducción}
\begin{center}
Un sistema de numeración es un conjunto de símbolos y reglas de generación que permiten construir todos los números válidos.
\end{center}

\section{Objetivos de la actividad}
\begin{center}
• Implementar dos funciones para la conversión de los sistemas de numeración.
\end{center}

\section{Metodología}
• Diseña e implementa dos funciones para la conversión entre sistemas de numeración en lenguaje C, donde la declaración de dichas funciones es la siguiente: \\

uint8_t base_to(uint8_t *binary_number, uint8_t k, const char *base_n_string, unsigned int binary_length) \\
void to_base(char *base_n_string, const uint8_t *binary_number, uint8_t k, unsigned int binary_length)\\

\\ Siendo base_to una función que permite convertir una cadena de texto base_n_string que contiene los dígitos de la base B = 2^k (k es el número de bits: 1 a 6) a un número binario bynary_number de longitud máxima binary_length, y regresa el número de bytes que contiene el número binario. Mientras que to_base es una función que permite convertir un número binario binary_number de longitud binary_length a una cadena de texto base_n_string, la cual contiene los dígitos de la base B = 2^k (k es el número de bits: 1 a 6).

\section{Contenido}
\subsection{Código:}
#include<stdlib.h>
#include<conio.h>
#include<stdio.h>
#include<stdint.h>

int num;
int opc;

void base_to(uint8_t *binary_number, uint8_t k, const char *base_n_string, unsigned int binary_length){
	}void binario() 
	{ 
	   int aux; 
	   if(num==0) 
	      return; 
	
	   aux=num%2; 
	   num=num/2; 
	   binario(); 

	   printf(" %d",aux); 	 
	}
	void octal(){
	    int aux1; 
	   if(num==0) 
	      return; 
	
	   aux1=num%8; 
	   num=num/8; 
	   octal(); 
	
	   printf(" %d",aux1); 
	}
	
	void hex(){
	 printf("\n");
	 printf(" %x ",num);
	}

int main() { 
	do{	
	   printf("\t\t < PRACTICA 1 >\n");
	   printf("\nConvertidor de numeros a distintas opciones:");
	   printf("\n Introduce un numero: "); 
	   scanf("%d",&num);
   
	   printf("\n 1- Binario \n 2- Octal \n 3- Hexadecimal \n 4- Salir \n Ingresa una opcion: ");
	   scanf("%d",&opc);
	   switch (opc)
	   {
	    case 1:
	    	printf("\nEl valor en binario es: \n");
			binario(); 
			printf("\n\n");
			system("pause");
			system("cls");
			break;
	 
	 	case 2: 
	 		printf("\nEl valor en octal es: \n");
			octal();
			printf("\n\n");
			system("pause");
			system("cls");
			break;
	 
		case 3: 
			printf("\nEl valor en hexadecimal es: \n");
			hex();
			printf("\n\n");
			system("pause");
			system("cls");
			break;
	 
	 	case 4:
	 		exit(4);
	 
		default:
			system("cls");
			printf("\n\n\tEsa opcion no esta disponible...\n");   
			system("pause");
			system("cls");
		}  
	} while ( opc = 5 );	   
}

\section{Resultados}
\begin{enumerate}
\item  EJERCICIO 1:\\
	\begin{center}
	\textbf{Respuesta: }  ------------------ MARCO ------------------ \\ 7        	//ALTO \\ 80       	//ANCHO \\ 2 - //restamos 2 caracteres a la pila (las orillas superiores) \\ swap \\ 2 - //restamos 2 caracteres a la pila (las orillas inferiores) \\ swap \\ "+" S. //Ingresamos la orilla superior izquierda \\ dup \\ 2 / \\ 8 - \\ dup \\ while dup 0 > //ciclo para hacer linea superior \\  "-" S. \\  1 - \\ drop \\ "¦" S. //simbolo antes de finalizar la palabra 'principal' " MenuPrincipal " S. //palabras puestas de 'titulo' \\ "+" S. //simbolo antes de comenzar la palabra 'menu' \\ dup \\ while dup 0 > //ciclo para hacer linea inferior \\  "-" S. \\  1 - \\ drop \\ swap \\ "+" S. //Ingresamos la orilla superior derecha \\ cr \\ swap \\ drop \\ swap \\ dup \\ swap \\ rot \\ swap \\ while dup 0 > //ciclo para hacer lineas de los costados \\   "¦" S. //primer linea izquierda del primer renglón \\   swap \\   dup \\   while dup 0 > \\   " " S. //Imprimir espacios \\   1 - \\   drop \\   swap \\   1 - \\   "¦" S. //Linea final del primer renglón \\   cr \\ drop \\ "+" S. //Ingresamos la orilla inferior izquierda \\ dup \\ while dup 0 > //ciclo de la linea final \\  "-" S. 1 - \\ swap \\ "+" S. //Ingresamos la orilla inferior derecha \\ swap \\ drop \\ cr \\ drop \\ drop \\ //------------ IMPRIMIR MENU -------------------- \\  //iniciamos en 3 en la pila y de ahi imprimimos todas las opciones, en cada uno de los siguientes if se hará lo mismo mientras vas bajando o subiendo en el menú se seguirá impriendo el menú. (termina la programacion de menu hasta que sale el siguiente comando //------ :*) \\ 3  \\ repeat  \\ if dup 3 = \\ 3 \\ 2 \\ OS:gotoxy //ubicamos en la posicion dada anteriormente \\ "ESC[7m" S. \\ "1. Imprime del 0-9" S. \\ "ESC[0m" S. \\ 3 \\ 3 \\ OS:gotoxy \\ "2. Imprime del -9-0" S. \\ 3 \\ 4 \\ OS:gotoxy \\ "3. Imprime de la a-z" S. \\ 3 \\ 5 \\ OS:gotoxy \\ "4. Imprime de la A-Z" S. \\ 3 \\ 6 \\ OS:gotoxy \\ "5. Salir" S. \\ //pos 4 en la pila \\ elif dup 4 = \\ 3 \\ 2 \\ OS:gotoxy \\ "1. Imprime del 0-9" S. \\ 3 \\ 3 \\ OS:gotoxy \\ "ESC[7m" S. \\
	
"2. Imprime del -9-0" S. \\ "ESC[0m" S. \\ 3 \\ 4 \\ OS:gotoxy \\ "3. Imprime de la a-z" S. \\ 3 \\ 5 \\ OS:gotoxy \\ "4. Imprime de la A-Z" S. \\ 3 \\ 6 \\ OS:gotoxy \\ "5. Salir" S. \\ //pos 5 en la pila \\ elif dup 5 = \\ 3 \\ 2 \\ OS:gotoxy \\ "1. Imprime del 0-9" S. \\ 3 \\ 3 \\ OS:gotoxy \\ "2. Imprime del -9-0" S. \\ 3 \\ 4 \\ OS:gotoxy \\ "ESC[7m" S. \\ "3. Imprime de la a-z" S. \\ "ESC[0m" S. \\ 3 \\ 5 \\ OS:gotoxy \\ "4. Imprime de la A-Z" S. \\ 3 \\ 6 \\ OS:gotoxy \\ "5. Salir" S. \\ //pos 6 en la pila \\ elif dup 6 = \\ 3 \\ 2 \\ OS:gotoxy \\ "1. Imprime del 0-9" S. \\ 3 \\ 3 \\ OS:gotoxy \\ "2. Imprime del -9-0" S. \\ 3 \\ 4 \\ OS:gotoxy \\ "3. Imprime de la a-z" S. \\ 3 \\ 5 

OS:gotoxy \\ "ESC[7m" S. \\ "4. Imprime de la A-Z" S. \\ "ESC[0m" S. \\ 3 \\ 6 \\ OS:gotoxy \\ "5. Salir" S. \\ //7 en la pila (ESC) \\ elif dup 7 = \\ 3 \\ 2 \\ OS:gotoxy \\ "1. Imprime del 0-9" S. \\ 3 \\ 3 \\ OS:gotoxy \\ "2. Imprime del -9-0" S. \\ 3 \\ 4 \\ OS:gotoxy \\ "3. Imprime de la a-z" S. \\ 3 \\ 5 \\ OS:gotoxy \\ "4. Imprime de la A-Z" S. \\ 3 \\ 6 \\ OS:gotoxy \\ "ESC[7m" S. \\ "5. Salir" S. \\ "ESC[0m" S. \\ //------- :* \\ //------RESTRICCIONES --------- \\ elif dup 1 = \\ 7 + \\ elif dup 2 = \\ 6 + \\ elif dup 8 = \\ 6 - \\ elif dup 9 = \\ 7 - \\ //---------------- TECLAS -------------- \\ //en los siguientes if moveremos la posicion dependiendo la flecha que eligas... \\ OS:key \\ if dup 1792834 = \\ swap \\ 1 + \\ swap \\ drop \\ elif dup 1792833 = \\ swap \\ 1 - \\ swap \\ drop \\ elif dup 1792835 = \\ swap \\ 2 + \\ swap \\ drop \\ elif dup 1792836 = \\ swap \\ 2 - \\ swap \\ drop \\  //--------- EJERCICIOS ------------ \\ //aqui inician los ejercicios que si al dar enter haces la opcion solicitada \\ elif dup 10 = \\ drop \\ cr cr \\ //opcion 1 \\ if dup 3 = \\ 0 . \\ 0 \\ dup \\ while 9 < \\  1 + \\ dup . \\ dup \\ drop \\ //opcion 2 \\ elif dup 4 = \\ cr \\ -9 \\ while dup \\  dup S. \\ ++ \\ 0 . \\ drop \\ //opcion 3 \\ elif dup 5 = \\ cr cr \\ for 97 upto 122 \\ dup emit \\ //opcion 4 \\ elif dup 6 = \\ cr cr cr \\ 65 \\ repeat \\ dup emit \\ ++ \\ dup \\ until 91 = \\  end \\ drop \\ //opcion 5 \\ elif dup 7 = \\ jump final \\ until dup 27 = end \\ final: cr cr cr cr 
	\end{center}
\end{enumerate}

\section{Conclusiones}  
En conclusión de esta tarea puedo decir que estos ejercicios ya fueron complicados ya que su complejidad ya era mucho mas alta, se puede notar viendo el tamaño de codigo que se hizo muy extenso pero con mis conocimientos y pidiendo algo de ayuda a los compañeros pude realizarlos correctamente. Es bueno aprender un lenguaje nuevo ya que yo ni si quiera había utilizado Linux ni mucho menos Latex, así que es una buena experiencia.

\section*{Agradecimientos}
Quiero hacer agradecimiento a mi profesor por explicarme cuando tenia dudas sobre cómo hacer los ejercicios, a mis compañeros porque varias veces me brindaron ayuda cuando tenia problemas y a mis padres en apoyarme cuando los necesito.

\begin{thebibliography}{00}
\bibitem{Alvarez2022} Becerra Alvarez, E. C. (2022, 4 octubre). ForEmb. https://drive.google.com/file/d/1hmdyOIhfwLkDTMcB8XZV8USGuOheWac2/view
\end{thebibliography}

\end{document}